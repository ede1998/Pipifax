\title{Pipifax Language Specification}
\author{
        Erik Hennig \\
                Dualer Student Informationstechnik\\
        Duale Hochschule Baden-W\"urttemberg\\
        Stuttgart \underline{Germany}
}
\date{\today}

\documentclass[a4paper, 12pt, english]{article}
\usepackage{hyperref}
\usepackage{listings}

\begin{document}
\maketitle
\thispagestyle{empty}
\newpage

\tableofcontents
\pagebreak


\section{Introduction}
Pipifax is a language designed for learning purposes. It exists, so students can learn to write a compiler for it. The original specification was published in German and can be found at \url{http://compilerbau.vavoor.de/?file=4-semester/pipifax-die-sprache} \\
This language specification is based on the original specification and includes additional features into the language.\\

Placeholder: \verb|$placeholder$|
\section{Basic structure}\label{basic structure}
A Pipifax program is divided into two major parts.\\
It starts with a number of include statements. Each statement adds an additional source file that is neccessary to compile the program.\\
The second part is the program itself. It can contain variables declarations, struct declarations and function declarations.

\subsection{Include statements}
Include statements look like this:
\begin{lstlisting}
<include "$path_to_program$" >
\end{lstlisting}
\verb|$path_to_program$| is any valid path to a file in the filesystem. It does not matter whether the path is relative or absolute.
When compiling, all includes are compiled (if found). If these includes have any dependencies, those are compiled as well until all dependencies are resolved.
\section{Types}\label{types}


\section{Conclusions}\label{conclusions}
We worked hard, and achieved very little.

\bibliographystyle{abbrv}
\bibliography{main}

\end{document}